\documentclass[14pt,compress]{beamer}
\usepackage[utf8]{inputenc}
\usepackage[T1]{fontenc}
\usepackage{lmodern}
\usepackage{tikz, upgreek}
\usetikzlibrary{patterns}
\usepackage{tikz-cd}
\usepackage{pifont}
%\tikzset{commutative diagrams/arrow style=math font}


% Beamer commands
\title{Inductive Types}
\date{}
%\setbeamertemplate{navigation symbols}{}
%\setbeamertemplate{blocks}[rounded][shadow=false]
%\usecolortheme{orchid}
% TikZ options
\usetikzlibrary{arrows,shapes}
\useoutertheme{split}
\useoutertheme[footline=authortitle]{miniframes}
%\useinnertheme{circles}
%\usecolortheme{whale}
%\usecolortheme{orchid}

\definecolor{beamer@blendedblue}{rgb}{0.3,0.5,0.8}
\definecolor{beamer@myviolet}{rgb}{0.7,0.2,0.5}
\definecolor{beamer@deepblue}{rgb}{0.5,0.5,0.7}
\definecolor{beamer@lightgray}{rgb}{0.5,0.7,0.5}
\definecolor{beamer@mybrown}{rgb}{0.3,0.3,0.2}
\definecolor{beamer@mathtext}{rgb}{0.9,0.5,0.4}
\definecolor{beamer@header}{rgb}{0.4,0.1,0.1}

\setbeamercolor{background canvas}{fg=white, bg=black}
\setbeamercolor{normal text}{fg=beamer@lightgray,bg=black}
\setbeamercolor{alerted text}{fg=red}
\setbeamercolor{example text}{fg=green!50!black}
\setbeamercolor{miniframes}{fg=red,bg=white}
\setbeamercolor{structure}{fg=beamer@deepblue}
\setbeamercolor{titlelike}{fg=magenta}
\setbeamercolor{frametitle}{fg=beamer@myviolet}
\setbeamercolor{title}{fg=beamer@myviolet}
\setbeamercolor{item}{fg=beamer@mybrown}
\setbeamercolor{section in head/foot}{fg=white,bg=beamer@header}

\setbeamerfont{framesubtitle}{size=10pt}


\newcommand{\N}{\mathbb{N}}
\newcommand{\tree}{\mathcal{BT}}
\newcommand{\U}{\mathcal{U}}

\mode
<all>

%\setbeamercovered{invisible}
\begin{document}

%%%%%%%%%%%%%%%%%%%%%%%%%%%%%%%%%%%%%%%%%%%%%%%%%%%%%%%%%%%%%%%%%%%%%%%%%%%%%%%%%%%%%%%%%%
\begin{frame}\label{frame : titlepage}
\titlepage
\end{frame}

%%%%%%%%%%%%%%%%%%%%%%%%%%%%%%%%%%%%%%%%%%%%%%%%%%%%%%%%%%%%%%%%%%%%%%%%%%%%%%%%%%%%%%%%%%
\section{Recap}
\begin{frame}\label{frame : recap}
\frametitle{Recap}
\begin{enumerate}
\item Untyped $\lambda$-calculus
\item Simyply typed $\lambda$-calculus
\item Unit type and product type
\end{enumerate}
\end{frame}

%%%%%%%%%%%%%%%%%%%%%%%%%%%%%%%%%%%%%%%%%%%%%%%%%%%%%%%%%%%%%%%%%%%%%%%%%%%%%%%%%%%%%%%%%%
\begin{frame}\label{frame : judgemental equality}
\frametitle{Judgemental Equality}

In addition to judgements of the form $a : A$, we now have a new kind of judgement
\textcolor{beamer@mathtext}{
\[ a \equiv b : A \]}
\noindent which is read as
\textcolor{beamer@mathtext}{``$a$ and $b$ are judgementally equal terms of type $A$''}.

\end{frame}

%%%%%%%%%%%%%%%%%%%%%%%%%%%%%%%%%%%%%%%%%%%%%%%%%%%%%%%%%%%%%%%%%%%%%%%%%%%%%%%%%%%%%%%%%%
\section{Inductive Types}
\begin{frame}\label{frame : gen of Nat}
\frametitle{The Type $\mathbb{N}$}


\begin{block}{Generators}
\textcolor{beamer@mathtext}{
\[ Z : \N \]
\[ succ : \N \to \N \]}
\end{block}
\end{frame}

%%%%%%%%%%%%%%%%%%%%%%%%%%%%%%%%%%%%%%%%%%%%%%%%%%%%%%%%%%%%%%%%%%%%%%%%%%%%%%%%%%%%%%%%%%
\begin{frame}\label{frame : rec of Nat}
\frametitle{The Type $\mathbb{N}$}

\begin{block}{Recursor}
\textcolor{beamer@mathtext}{
\[ rec_{\N}^C : C \to (\N \to C \to C) \to (\N \to C) \]
\[ rec_{\N}^C (c_0,c_s,Z) :\equiv c_0 \]
\[ rec_{\N}^C (c_0,c_s,(succ(n))) :\equiv c_s(n, rec_{\N}^C(c_0,c_s,n)) \]}
\end{block}
\end{frame}

%%%%%%%%%%%%%%%%%%%%%%%%%%%%%%%%%%%%%%%%%%%%%%%%%%%%%%%%%%%%%%%%%%%%%%%%%%%%%%%%%%%%%%%%%%
\begin{frame}\label{frame : gen of co-product}
\frametitle{Co-Product of Types}

\begin{block}{Generators}
\textcolor{beamer@mathtext}{
\[ inl : A \to (A + B) \]
\[ inr : B \to (A + B) \]}
\end{block}
\end{frame}

%%%%%%%%%%%%%%%%%%%%%%%%%%%%%%%%%%%%%%%%%%%%%%%%%%%%%%%%%%%%%%%%%%%%%%%%%%%%%%%%%%%%%%%%%%
\begin{frame}\label{frame : rec of co-product}
\frametitle{Co-Product of Types}

\begin{block}{Recursor}
\textcolor{beamer@mathtext}{
\[ rec_{A+B}^C : (A \to C) \to (B \to C) \to ((A + B) \to C) \]
\[ rec_{A+B}^C(f,g,inl(a)) :\equiv f(a) \]
\[ rec_{A+B}^C(f,g,inr(b)) :\equiv g(b) \]}
\end{block}
\end{frame}

%%%%%%%%%%%%%%%%%%%%%%%%%%%%%%%%%%%%%%%%%%%%%%%%%%%%%%%%%%%%%%%%%%%%%%%%%%%%%%%%%%%%%%%%%%
\begin{frame}\label{frame : product type}
\frametitle{Product of Types}

\begin{block}{Generator}
\textcolor{beamer@mathtext}{
\[ (\_,\_) : A \to B \to (A \times B) \]}
\end{block}
\pause
\begin{block}{Recursor}
\textcolor{beamer@mathtext}{
\[ rec_{A \times B}^C : (A \to B \to C) \to ((A \times B) \to C) \]
\[ rec_{A \times B}^C(f,(a,b)) :\equiv f(a)(b) \]}
\end{block}
\end{frame}

%%%%%%%%%%%%%%%%%%%%%%%%%%%%%%%%%%%%%%%%%%%%%%%%%%%%%%%%%%%%%%%%%%%%%%%%%%%%%%%%%%%%%%%%%%
\begin{frame}\label{frame : projections}
\frametitle{Product of Types}

\begin{block}{Projections}
\textcolor{beamer@mathtext}{
\[ \pi_1 :\equiv rec_{A \times B}^A (\lambda a. \lambda b. a) : (A \times B) \to A\]
\[ \pi_2 :\equiv rec_{A \times B}^B (\lambda a. \lambda b. b) : (A \times B) \to B\]}
\end{block}

\pause
\begin{block}{}
Note that $\pi_1((a,b)) \equiv a$ and $\pi_2((a,b)) \equiv b$
follows from the definition of the recursor.
\end{block}

\end{frame}

%%%%%%%%%%%%%%%%%%%%%%%%%%%%%%%%%%%%%%%%%%%%%%%%%%%%%%%%%%%%%%%%%%%%%%%%%%%%%%%%%%%%%%%%%%
\begin{frame}\label{frame : unit type}
\frametitle{The Unit Type}

\begin{block}{Generator}
\textcolor{beamer@mathtext}{
\[ \star : \mathbf{1} \]}
\end{block}
\pause
\begin{block}{Recursor}
\textcolor{beamer@mathtext}{
\[ rec_{\mathbf{1}}^C : C \to (\mathbf{1} \to C) \]
\[ rec_{\mathbf{1}}^C(c,\star) :\equiv c \]}
\end{block}
\end{frame}

%%%%%%%%%%%%%%%%%%%%%%%%%%%%%%%%%%%%%%%%%%%%%%%%%%%%%%%%%%%%%%%%%%%%%%%%%%%%%%%%%%%%%%%%%%
\begin{frame}\label{frame : void type}
\frametitle{The Void Type}

\begin{block}{Generator}
\textcolor{beamer@mathtext}{
None!}
\end{block}
\pause
\begin{block}{Recursor}
\textcolor{beamer@mathtext}{
\[ rec_{\mathbf{0}}^C : \mathbf{0} \to C \]}
\end{block}
\end{frame}


%%%%%%%%%%%%%%%%%%%%%%%%%%%%%%%%%%%%%%%%%%%%%%%%%%%%%%%%%%%%%%%%%%%%%%%%%%%%%%%%%%%%%%%%%%
\begin{frame}\label{frame : gen of tree}
\frametitle{The Type of Binary Trees}

\begin{block}{Generators}
\textcolor{beamer@mathtext}{
\[ leaf : A \to \tree(A,B) \]
\[ node : B \to \tree(A,B) \to \tree(A,B) \to \tree(A,B) \]}
\end{block}
\end{frame}

%%%%%%%%%%%%%%%%%%%%%%%%%%%%%%%%%%%%%%%%%%%%%%%%%%%%%%%%%%%%%%%%%%%%%%%%%%%%%%%%%%%%%%%%%%
\begin{frame}\label{frame : rec of tree}
\frametitle{The Type of Binary Trees}

\begin{block}{Recursor}
$ $\\
\pause
\begin{huge}
Homework!!
\end{huge}

\end{block}
\end{frame}

%%%%%%%%%%%%%%%%%%%%%%%%%%%%%%%%%%%%%%%%%%%%%%%%%%%%%%%%%%%%%%%%%%%%%%%%%%%%%%%%%%%%%%%%%%
\section{Proposition as Types}
\begin{frame}\label{frame : checkpoint}
\frametitle{Checkpoint Reached!!}

\begin{enumerate}
\pause
\item
Type theories as programming languages.
\pause
\textcolor{beamer@mathtext}
{\begin{Large} \ding{51} \end{Large}}
\pause
\item
Type theories as foundational systems.
\textcolor{beamer@mathtext}
{\begin{Large} ? \end{Large}}
\end{enumerate}
\end{frame}
%%%%%%%%%%%%%%%%%%%%%%%%%%%%%%%%%%%%%%%%%%%%%%%%%%%%%%%%%%%%%%%%%%%%%%%%%%%%%%%%%%%%%%%%%%
\begin{frame}\label{frame : prop as types}
\frametitle{Propositions as Types}

\pause
A proposition can be considered as a type $A$ with 
the witnesses/proofs as its terms.
\pause
So to prove $A$ we need to show that 
there is a term $a : A$.

\pause
\color{beamer@mathtext}
\begin{table}
\begin{tabular}{ c | c }

\hline
Propositions & Types\\ 
\hline

True & $\mathbf{1}$ \\

False & $\mathbf{0}$ \\

$A \wedge B$ & $A \times B$ \\

$A \vee B$ & $A + B$ \\

$A \implies B$ & $A \to B$ \\

$\neg A$ & $A \to \mathbf{0}$ \\
\end{tabular}
\end{table}
\end{frame}

%%%%%%%%%%%%%%%%%%%%%%%%%%%%%%%%%%%%%%%%%%%%%%%%%%%%%%%%%%%%%%%%%%%%%%%%%%%%%%%%%%%%%%%%%%
\begin{frame}\label{frame : intuitionism}
\frametitle{Digression: Intuitionism}
\pause
Type theories are better equiped to model
\textcolor{beamer@mathtext}{intuitionistic/constructive logic}
rather than \textcolor{beamer@mathtext}{classical logic}.
\pause
Classical axioms such as
\textcolor{beamer@mathtext}{double negation elimination($\neg \neg A \implies A$)} and
\textcolor{beamer@mathtext}{excluded middle($A \vee \neg A$)}
are NOT provable in propositions as types interpretation.

\end{frame}

%%%%%%%%%%%%%%%%%%%%%%%%%%%%%%%%%%%%%%%%%%%%%%%%%%%%%%%%%%%%%%%%%%%%%%%%%%%%%%%%%%%%%%%%%%
\begin{frame}\label{frame : missing pieces}
\frametitle{Missing Pieces}
\begin{enumerate}
\item Quantifiers(first order and higher order)
\item Equality
\end{enumerate}
\end{frame}

%%%%%%%%%%%%%%%%%%%%%%%%%%%%%%%%%%%%%%%%%%%%%%%%%%%%%%%%%%%%%%%%%%%%%%%%%%%%%%%%%%%%%%%%%%
\section{Dependent Types}
\begin{frame}\label{frame : dependent types}
\frametitle{Dependent Types}

We will extend type theory such that: 
\begin{enumerate}
\item Types can depend on terms.
\item Types can be terms themselves.
\end{enumerate}
\end{frame}

%%%%%%%%%%%%%%%%%%%%%%%%%%%%%%%%%%%%%%%%%%%%%%%%%%%%%%%%%%%%%%%%%%%%%%%%%%%%%%%%%%%%%%%%%%
\begin{frame}\label{frame : universes}

\frametitle{Universes}
\pause
To consider types are terms we need types of types. These special types
whose terms are types are called \textcolor{beamer@mathtext}{universes}.
We consider a cumulative sequence of universes,
\textcolor{beamer@mathtext}{
\[ \U_0 : \U_1 : \U_2 :\dots \]}
\end{frame}

%%%%%%%%%%%%%%%%%%%%%%%%%%%%%%%%%%%%%%%%%%%%%%%%%%%%%%%%%%%%%%%%%%%%%%%%%%%%%%%%%%%%%%%%%%
\begin{frame}\label{frame : universes and type families}
\frametitle{Type Families}
Given a type $A$ and universe $\U$ a function $f : A \to \U$ is called a
\textcolor{beamer@mathtext}{type family}.

\end{frame}

%%%%%%%%%%%%%%%%%%%%%%%%%%%%%%%%%%%%%%%%%%%%%%%%%%%%%%%%%%%%%%%%%%%%%%%%%%%%%%%%%%%%%%%%%%
\begin{frame}\label{frame : dependent function type}
\frametitle{Dependent Function/Product Type}
Add that they are also called dependent product type and say the reason
\end{frame}

%%%%%%%%%%%%%%%%%%%%%%%%%%%%%%%%%%%%%%%%%%%%%%%%%%%%%%%%%%%%%%%%%%%%%%%%%%%%%%%%%%%%%%%%%%
\begin{frame}\label{frame : dependent sum type}
\frametitle{Dependent Sum Type}

\end{frame}

%%%%%%%%%%%%%%%%%%%%%%%%%%%%%%%%%%%%%%%%%%%%%%%%%%%%%%%%%%%%%%%%%%%%%%%%%%%%%%%%%%%%%%%%%%
\end{document}


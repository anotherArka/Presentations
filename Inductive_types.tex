\documentclass[14pt,compress]{beamer}
\usepackage[utf8]{inputenc}
\usepackage[T1]{fontenc}
\usepackage{lmodern}
\usepackage{tikz, upgreek}
\usetikzlibrary{patterns}
\usepackage{tikz-cd}
%\tikzset{commutative diagrams/arrow style=math font}

% Beamer commands
\title{Inductive Types}
\date{}
%\setbeamertemplate{navigation symbols}{}
%\setbeamertemplate{blocks}[rounded][shadow=false]
%\usecolortheme{orchid}
% TikZ options
\usetikzlibrary{arrows,shapes}
\useoutertheme{split}
\useoutertheme[footline=authortitle]{miniframes}
%\useinnertheme{circles}
%\usecolortheme{whale}
%\usecolortheme{orchid}

\definecolor{beamer@blendedblue}{rgb}{0.3,0.5,0.8}
\definecolor{beamer@myviolet}{rgb}{0.7,0.2,0.5}
\definecolor{beamer@deepblue}{rgb}{0.5,0.5,0.7}
\definecolor{beamer@lightgray}{rgb}{0.5,0.7,0.5}
\definecolor{beamer@mybrown}{rgb}{0.3,0.3,0.2}
\definecolor{beamer@mathtext}{rgb}{0.9,0.5,0.4}
\definecolor{beamer@header}{rgb}{0.4,0.1,0.1}

\setbeamercolor{background canvas}{fg=white, bg=black}
\setbeamercolor{normal text}{fg=beamer@lightgray,bg=black}
\setbeamercolor{alerted text}{fg=red}
\setbeamercolor{example text}{fg=green!50!black}
\setbeamercolor{miniframes}{fg=red,bg=white}
\setbeamercolor{structure}{fg=beamer@deepblue}
\setbeamercolor{titlelike}{fg=magenta}
\setbeamercolor{frametitle}{fg=beamer@myviolet}
\setbeamercolor{title}{fg=beamer@myviolet}
\setbeamercolor{item}{fg=beamer@mybrown}
\setbeamercolor{section in head/foot}{fg=white,bg=beamer@header}

\setbeamerfont{framesubtitle}{size=10pt}


\newcommand{\N}{\mathbb{N}}

\mode
<all>

%\setbeamercovered{invisible}
\begin{document}

%%%%%%%%%%%%%%%%%%%%%%%%%%%%%%%%%%%%%%%%%%%%%%%%%%%%%%%%%%%%%%%%%%%%%%%%%%%%%%%%%%%%%%%%%%
\begin{frame}\label{frame : titlepage}
\titlepage
\end{frame}

%%%%%%%%%%%%%%%%%%%%%%%%%%%%%%%%%%%%%%%%%%%%%%%%%%%%%%%%%%%%%%%%%%%%%%%%%%%%%%%%%%%%%%%%%%
\begin{frame}\label{frame : judgemental equality}
\frametitle{Judgemental Equality}

In addition to judgements of the form $a : A$, we now have a new kind of judgement
\textcolor{beamer@mathtext}{
\[ a \equiv b : A \]}
\noindent which is read as
\textcolor{beamer@mathtext}{``$a$ and $b$ are judgementally equal terms of type $A$''}.

\end{frame}

%%%%%%%%%%%%%%%%%%%%%%%%%%%%%%%%%%%%%%%%%%%%%%%%%%%%%%%%%%%%%%%%%%%%%%%%%%%%%%%%%%%%%%%%%%
\begin{frame}\label{frame : gen of Nat}
\frametitle{The Type $\mathbb{N}$}

\begin{block}{Generators}
\[ Z : \N \]
\[ succ : \N \to \N \]
\end{block}
\end{frame}

%%%%%%%%%%%%%%%%%%%%%%%%%%%%%%%%%%%%%%%%%%%%%%%%%%%%%%%%%%%%%%%%%%%%%%%%%%%%%%%%%%%%%%%%%%
\begin{frame}\label{frame : rec of Nat}
\frametitle{The Type $\mathbb{N}$}

\begin{block}{Recursor}
\[ rec_{\N,C} : C \to (\N \to C \to C) \to (\N \to C) \]
\[ rec_{\N,C}(c_0,c_s,Z) \equiv c_0 \]
\[ rec_{\N,C}(c_0,c_s,(succ(n))) \equiv c_s(n, rec_{\N,C}(c_0,c_s,n)) \]
\end{block}
\end{frame}

%%%%%%%%%%%%%%%%%%%%%%%%%%%%%%%%%%%%%%%%%%%%%%%%%%%%%%%%%%%%%%%%%%%%%%%%%%%%%%%%%%%%%%%%%%
\begin{frame}\label{frame : gen of co-product}
\frametitle{Co-Product of Types}

\begin{block}{Generators}
\[ inl : A \to (A + B) \]
\[ inr : B \to (A + B) \]
\end{block}
\end{frame}

%%%%%%%%%%%%%%%%%%%%%%%%%%%%%%%%%%%%%%%%%%%%%%%%%%%%%%%%%%%%%%%%%%%%%%%%%%%%%%%%%%%%%%%%%%
\begin{frame}\label{frame : rec of co-product}
\frametitle{Co-Product of Types}

\begin{block}{Recursor}
\[ rec_{A+B,C} : (A \to C) \to (B \to C) \to ((A + B) \to C) \]
\[ rec_{A+B,C}(f,g,inl(a)) \equiv f(a) \]
\[ rec_{A+B,C}(f,g,inr(b)) \equiv g(b) \]
\end{block}
\end{frame}

%%%%%%%%%%%%%%%%%%%%%%%%%%%%%%%%%%%%%%%%%%%%%%%%%%%%%%%%%%%%%%%%%%%%%%%%%%%%%%%%%%%%%%%%%%

\end{document}


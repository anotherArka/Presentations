\documentclass[14pt,compress]{beamer}
\usepackage[utf8]{inputenc}
\usepackage[T1]{fontenc}
\usepackage{lmodern}
\usepackage{tikz, upgreek}
\usetikzlibrary{patterns}
\usepackage{tikz-cd}
%\tikzset{commutative diagrams/arrow style=math font}

% Beamer commands
\title{Final Talk}
\date{}
%\setbeamertemplate{navigation symbols}{}
%\setbeamertemplate{blocks}[rounded][shadow=false]
%\usecolortheme{orchid}
% TikZ options
\usetikzlibrary{arrows,shapes}
\useoutertheme{split}
\useoutertheme[footline=authortitle]{miniframes}
%\useinnertheme{circles}
%\usecolortheme{whale}
%\usecolortheme{orchid}

\definecolor{beamer@blendedblue}{rgb}{0.3,0.5,0.8}
\definecolor{beamer@myviolet}{rgb}{0.7,0.2,0.5}
\definecolor{beamer@deepblue}{rgb}{0.5,0.5,0.7}
\definecolor{beamer@lightgray}{rgb}{0.5,0.7,0.5}
\definecolor{beamer@mybrown}{rgb}{0.3,0.3,0.2}
\definecolor{beamer@mathtext}{rgb}{0.9,0.5,0.4}
\definecolor{beamer@header}{rgb}{0.4,0.1,0.1}

\setbeamercolor{background canvas}{fg=white, bg=black}
\setbeamercolor{normal text}{fg=beamer@lightgray,bg=black}
\setbeamercolor{alerted text}{fg=red}
\setbeamercolor{example text}{fg=green!50!black}
\setbeamercolor{miniframes}{fg=red,bg=white}
\setbeamercolor{structure}{fg=beamer@deepblue}
\setbeamercolor{titlelike}{fg=magenta}
\setbeamercolor{frametitle}{fg=beamer@myviolet}
\setbeamercolor{title}{fg=beamer@myviolet}
\setbeamercolor{item}{fg=beamer@mybrown}
\setbeamercolor{section in head/foot}{fg=white,bg=beamer@header}

\setbeamerfont{framesubtitle}{size=10pt}


\mode
<all>

%\setbeamercovered{invisible}
\begin{document}

%%%%%%%%%%%%%%%%%%%%%%%%%%%%%%%%%%%%%%%%%%%%%%%%%%%%%%%%%%%%%%%%%%%%%%%%%%%%%%%%%%%%%%%%%%

\begin{frame}\label{frame : titlepage}
\titlepage
\end{frame}

%%%%%%%%%%%%%%%%%%%%%%%%%%%%%%%%%%%%%%%%%%%%%%%%%%%%%%%%%%%%%%%%%%%%%%%%%%%%%%%%%%%%%%%%%%
\section{Recap}

\begin{frame}\label{frame : recap}
\frametitle{Recap}

\begin{enumerate}
\item Lambda Calculus(Untyped and Typed)
\item Inductive Types and Recursion
\item Dependent Types
\item Equality Type
\end{enumerate}
\end{frame}

%%%%%%%%%%%%%%%%%%%%%%%%%%%%%%%%%%%%%%%%%%%%%%%%%%%%%%%%%%%%%%%%%%%%%%%%%%%%%%%%%%%%%%%%%%
\section{Univalence Axiom}

\begin{frame}

\frametitle{Univalence Axiom}

\[ f \sim g : \equiv \prod_{x : A} (f(x) = g(x) \]
\pause

\[ A  \simeq B : \sum_{f : A \to B} \sum_{g : B \to A}
  (g \circ f \sim id_B) \times (f \circ g \sim id_A) \]
\pause

\textcolor{beamer@mathtext}{
\[ (A \simeq B) \simeq (A = B) \] }
\end{frame}

%%%%%%%%%%%%%%%%%%%%%%%%%%%%%%%%%%%%%%%%%%%%%%%%%%%%%%%%%%%%%%%%%%%%%%%%%%%%%%%%%%%%%%%%%%
\begin{frame}
\frametitle{Cubical Type Theory}

A possible way to build a system where functional extensionality and
univalence can be proven.
The main idea is to add an interval type and its computational properties.
This is done in different ways in Agda, Cubicaltt, Redtt etc.

\end{frame}

%%%%%%%%%%%%%%%%%%%%%%%%%%%%%%%%%%%%%%%%%%%%%%%%%%%%%%%%%%%%%%%%%%%%%%%%%%%%%%%%%%%%%%%%%%

\end{document}

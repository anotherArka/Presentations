\documentclass[10pt]{beamer}
\usepackage{listings}
%\usetheme{Boadilla}
%\usecolortheme{beaver}

\usepackage[T1]{fontenc}
\usepackage{multicol}
\usepackage[latin9]{inputenc}
\usepackage{graphicx}
\usepackage{array}
%\usepackage{enumitem}
\usepackage{amsthm,amsmath,latexsym,amssymb,amsmath}
\usepackage{mathrsfs,graphicx,xargs,etoolbox}


\DeclareOptionBeamer{compress}{\beamer@compresstrue}
\ProcessOptionsBeamer

\mode<presentation>

\useoutertheme[footline=authortitle]{miniframes}
\useinnertheme{circles}
\usecolortheme{whale}
\usecolortheme{orchid}

\definecolor{beamer@blendedblue}{rgb}{0.137,0.466,0.741}

\setbeamercolor{structure}{fg=beamer@blendedblue}
\setbeamercolor{titlelike}{parent=structure}
\setbeamercolor{frametitle}{fg=black}
\setbeamercolor{title}{fg=black}
\setbeamercolor{item}{fg=black}

\mode
<all>

%commands
\newcommand{\U}{\mathscr{U}}
\newcommand{\R}{\mathbb{R}}

\theoremstyle{definition}
\newtheorem{question}{Question}[section]

\theoremstyle{definition}
\newtheorem{answer}{Answer}[section]


\setbeamercovered{invisible}

\title{J\' onsson\,-Tarski Theorem}
\author{Arka Ghosh}
\date{}

\begin{document}

\begin{frame}\label{frame: titlepage}
\titlepage
\end{frame}

%%%%%%%%%%%%%%%%%%%%%%%%%%%%%%%%%%%%%%%%%%%%%%%%%%%%%%%%%%%%%%%%%%%%%%%%%%%%%%%%%

\begin{frame}\label{frame: modal similarity type}

\visible<1->{
\begin{definition}[Modal similarity type]\label{defn: modal similarity type}
A modal similarity type $\tau$ is a pair $(O,\rho)$ where $O$ is a non-empty set, 
and $\rho$ is a function $O\to\mathbb{N}$. An element $\Delta \in O$ is called a modal
operator of arity $\rho(\Delta)$.
\end{definition}
}

\visible<2->{
\begin{definition}\label{defn: frame of a modal similarity type}
\end{definition}
Let $\tau$ be a modal similarity type. Let $W$ be a set and for every $\Delta \in \tau$ 
let $R_{\Delta}$ be a $(\rho(\Delta) + 1)$-ary relation on $W$.
Then $(W, R_{\Delta})_{\Delta \in \tau}$ is called a $\tau$-frame.
}

\end{frame}

%%%%%%%%%%%%%%%%%%%%%%%%%%%%%%%%%%%%%%%%%%%%%%%%%%%%%%%%%%%%%%%%%%%%%%%%%%%%%%%%%%%%

\begin{frame}\label{frame: boolean algebra}

\begin{definition}[Boolean algebra]\label{defn: boolean algebra}
Let $A$ be a non-empty set, $0 \in A$, $+$ be a binary operator and $-$ be an unary operator
on $A$. Define $1 := 0$ and let $x\cdot y$ be the abbreviation for $-((-x)+(-y))$.\smallskip

$\mathcal{K} = (A,+,-,0)$ is a called a boolean algebra if the following hold
for $x, y, z \in A$:
\begin{multicols}{2}
\begin{enumerate}
\item[$\blacktriangleright$] $-(-x) = x$
\item[$\blacktriangleright$] $x + 0 = x$
\item[$\blacktriangleright$] $x + 1 = 1$
\item[$\blacktriangleright$] $x + x = x$
\item[$\blacktriangleright$] $x + (y + z) = (x + y) + z$
\item[$\blacktriangleright$] $x + y = y + x$
\item[$\blacktriangleright$] $x + (-x) = 1$
\item[$\blacktriangleright$] $x + (y\cdot z) = (x + y) \cdot (x + z)$
\item[]
\item[$\blacktriangleright$] $x \cdot 0 = 0$
\item[$\blacktriangleright$] $x \cdot 1 = x$
\item[$\blacktriangleright$] $x \cdot x = x$
\item[$\blacktriangleright$] $x \cdot (y \cdot z) = (x \cdot y) \cdot z$
\item[$\blacktriangleright$] $x \cdot y = y \cdot x$
\item[$\blacktriangleright$] $x \cdot (-x) = 1$
\item[$\blacktriangleright$] $x \cdot (y + z) = (x \cdot y) + (x \cdot z)$
\end{enumerate}
\end{multicols}
\end{definition}
\end{frame}

%%%%%%%%%%%%%%%%%%%%%%%%%%%%%%%%%%%%%%%%%%%%%%%%%%%%%%%%%%%%%%%%%%%%%%%%%%%%%%%%%%%

\begin{frame}\label{frame: boolean algebra with operators}

\frametitle{Algebraizing modal logic}

\visible<2->{
\begin{block}{Axioms for the normal modal logic $\mathbf{K}$}
$\Diamond \bot \longleftrightarrow \bot$\\
$\Diamond (p \vee q) \longleftrightarrow \Diamond p \vee \Diamond q$
\end{block}
}

\visible<3->{
\begin{definition}[Boolean algebra with operators]
Let $\tau = (O, \rho)$ be a modal similarity type. Then
$\mathcal{J} = (A, +, -, 0, f_{\Delta})_{\Delta \in \tau}$ is called a boolean algebra
with operators(BAO) if the following holds:

\begin{enumerate}
\item[$\blacktriangleright$] $(A,+,-,0)$ is a boolean algebra.

\item[$\blacktriangleright$] For $\Delta \in O$, $f_{\Delta}$ 
is an operator of arity $\rho(\Delta)$.

\item[$\blacktriangleright$] (Normality) $f_{\Delta}(a_1,\dots,a_{\rho (\Delta)}) = 0$ 
if $a_i = 0$ for some $i$.

\item[$\blacktriangleright$] (Additivity) For all $i$,
$f_{\Delta}(a_1,\dots,a_i + a_i',\dots,a_n) = $
$$f_{\Delta}(a_1,\dots,a_i,\dots,a_n) + 
  f_{\Delta}(a_1,\dots,a_i',\dots,a_n) $$
}
  
\end{enumerate}
\end{definition}
\end{frame}

%%%%%%%%%%%%%%%%%%%%%%%%%%%%%%%%%%%%%%%%%%%%%%%%%%%%%%%%%%%%%%%%%%%%%%%%%%%%%%%%%%%%%%%%%%

\begin{frame} 

\frametitle{Algebraizing modal logic}

\begin{definition}[Homomorphisms between boolean algebras]
\end{definition}

\begin{definition}[Homomorphisms between BAOs]
\end{definition}


\end{frame}

%%%%%%%%%%%%%%%%%%%%%%%%%%%%%%%%%%%%%%%%%%%%%%%%%%%%%%%%%%%%%%%%%%%%%%%%%%%%%%%%%%%%%%%%%%

\begin{frame}\label{frame: complex algebra}
\frametitle{Algebraizing modal semantics}

\only<1>{
\begin{definition}\label{defn: alg modal op}
Let $\tau$ be a modal similarity type $\tau = (O, \rho)$ and 
$(W,R_{\Delta})_{\Delta \in \tau}$ a $\tau$-frame. For every $\Delta \in \tau$ define
the $\rho(\Delta)$-ary function $m_{\Delta}$ on $\mathcal{P}(W)$ as:
$$ m_{\Delta}(X_1,\dots,X_{\rho(\Delta)}) = $$ 
$$\left\lbrace x \in W\ |\ \exists x_1\in X_1,\dots,x_{\rho(\Delta)}\in X_{\rho(\Delta)} 
  \text{ s.t. } R_{\Delta}xx_1\dots x_{\rho(\Delta)} \right\rbrace$$
\end{definition}
}

\only<2->{
\begin{definition}[Complex algebra]

\begin{enumerate}
\visible<2->{
\item[$\blacktriangleright$] Let $\tau$ be a modal similarity type, 
and $\mathcal{M} = (W, R_{\Delta})_{\Delta \in \tau}$ a $\tau$-frame. 
The (full) complex algebra of $\mathcal{M}$ (notation: $\mathcal{M}^+$ ) is defined as
$$ \mathcal{M}^+ = (\mathcal{P}(W), \cup,(\cdot)^c, m_{\Delta})_{\Delta \in \tau},$$ 
where $(\mathcal{P}(W), \cup,(\cdot)^c)$ is the power-set algebra.
}

\visible<3->{
\item[$\blacktriangleright$] A complex algebra is a subalgebra of a full complex algebra.
}

\visible<4->{
\item[$\blacktriangleright$] If $\mathcal{F}$ is a class of frames, then we denote 
the class of full complex algebras of frames in $\mathcal{F}$ by $Cm \mathcal{F}$.
}
\end{enumerate}

 
\end{definition}
}

\end{frame}

%%%%%%%%%%%%%%%%%%%%%%%%%%%%%%%%%%%%%%%%%%%%%%%%%%%%%%%%%%%%%%%%%%%%%%%%%%%%%%%%%%%%%%%%

\begin{frame}\label{frame: filters and ultrafilters}

\frametitle{Filters and ultrafilters}

\begin{definition}[Filter]\label{defn: filter}
Given a boolean algebra $\mathcal{K} = (A,+,-,0)$, $F\subseteq A$ is called a filter if:
\begin{enumerate}
\item[$\blacktriangleright$] $1 \in F$,
\item[$\blacktriangleright$] If $a,b \in F$ then $a \cdot b \in F$,
\item[$\blacktriangleright$] If $a \in F$ and $a \leq b$ then $b \in F$.
\end{enumerate}
$F$ is called proper if $0 \notin F$, or, equivalently, if $F$ is a strict subset of $A$.
\end{definition}

\visible<2->{
\begin{definition}[Ultrafilter]\label{defn: ultrafilter}
A filter $F$ is called an ultrafilter if for every $a \in A$, either $a$ or $-a$ 
belongs to $F$. We will denote the collection of ultrafilters of $\mathcal{K}$ as
$\mathbf{UF}\mathcal{K}$. 
\end{definition}
}

\end{frame}

\begin{frame}\label{frame: examples of filters}
\frametitle{Examples of filters}
\visible<1->{
\begin{example}\label{example: filter generated by an element}
Given $a \in A$, the set $a\uparrow := \{b \in A\ |\ a \leq b\}$ is a filter.
\end{example}
}

\visible<2->{
\begin{example}\label{example: filter generated by a subset}
Given $D \subseteq A$ the filter generated by $D$ is defined as, 
 $$F_D := \{a \in A\ |\ \exists d_0,\dots,d_n \in D \text{ s.t. } 
   d_0\cdot \dots \cdot d_n \leq a \}.$$
This is also the smallest filter containing $D$.   
\end{example}
}

\end{frame}

%%%%%%%%%%%%%%%%%%%%%%%%%%%%%%%%%%%%%%%%%%%%%%%%%%%%%%%%%%%%%%%%%%%%%%%%%%%%%%%%%%%%%%%%%%%%%%

\begin{frame}\label{frame: finite meet property}

\visible<1->{
\begin{definition}[Finite meet property]\label{defn: finite meet property}
$D \subseteq A$ is said to have finite meet property if there is no finite subset 
$\{d_0,\dots,d_n\}$ of $D$ such that $d_0\cdot \dots \cdot d_n = 0$. 
\end{definition}
}

\visible<2->{
\begin{lemma}\label{lemma: when generated filter is proper}
Given $D \subseteq A$, $F_D$ is proper if and only if $D$ has the finite meet property
\end{lemma}
}

\end{frame}

%%%%%%%%%%%%%%%%%%%%%%%%%%%%%%%%%%%%%%%%%%%%%%%%%%%%%%%%%%%%%%%%%%%%%%%%%%%%%%%%%%%%%%%%%%%%%%

\begin{frame}\label{frame: prp of ultrafilters}

\frametitle{Properties of ultrafilters}

\visible<1->{
\begin{lemma}\label{lemma: ultrafilters are prime wrt +}
For and ultrafilter $F$, $a + b \in F$ iff $a \in F$ or $b \in F$.
\end{lemma}
}

\visible<2->{
\begin{theorem}\label{thm: ultrafilters are maximal}
Ultrafilters are also maximal filters. That is, if $F$ is an ultrafilter and $L$ is a proper
filter such that $F \subseteq L$ then $F = L$.
\end{theorem}
}

\visible<3->{
\begin{theorem}[Ultrafilter theorem]\label{thm: ultrafilter theorem}
Let $a \in A$, and $F$ is a proper filter of $A$, such that $a \notin F$. Then there is a
ultrafilter $L$, such that $F \subseteq L$ and $a \notin F$.
\end{theorem}
}
\end{frame}

%%%%%%%%%%%%%%%%%%%%%%%%%%%%%%%%%%%%%%%%%%%%%%%%%%%%%%%%%%%%%%%%%%%%%%%%%%%%%%%%%%%%%%%%%%%%%%

\begin{frame}\label{frame: Stone embedding}

\frametitle{Stone embedding}

\visible<1->{
\begin{definition}[Stone embedding]
Let $\mathcal{K} = (A,+,-,0)$ be a boolean algebra. The Stone embedding 
$\rho : A \to \mathcal{P}(\mathbb{UF}\mathcal{K})$ is defined as,
$$\rho(a) = \{V \in \mathbb{UF}\mathcal{K}\ |\ a\in V \}$$
\end{definition}
}

\visible<2->{
\begin{theorem}
$\rho$ is an embedding of boolean algebras.
\end{theorem}
}
\end{frame}

%%%%%%%%%%%%%%%%%%%%%%%%%%%%%%%%%%%%%%%%%%%%%%%%%%%%%%%%%%%%%%%%%%%%%%%%%%%%%%%%%%%%%%%%%%%%

\begin{frame}
\frametitle{Stone embedding}

\begin{proof}
$\rho$ is a homomorphism
\end{proof}

\end{frame}

\end{document}
\documentclass[10pt]{beamer}
\usepackage{listings}
%\usetheme{Boadilla}
%\usecolortheme{beaver}

\usepackage[T1]{fontenc}
\usepackage{multicol}
\usepackage[latin9]{inputenc}
\usepackage{graphicx}
\usepackage{array}
%\usepackage{enumitem}
\usepackage{amsthm,amsmath,latexsym,amssymb,amsmath}
\usepackage{mathrsfs,graphicx,xargs,etoolbox}


\DeclareOptionBeamer{compress}{\beamer@compresstrue}
\ProcessOptionsBeamer

\mode<presentation>

\useoutertheme[footline=authortitle]{miniframes}
\useinnertheme{circles}
\usecolortheme{whale}
\usecolortheme{orchid}

\definecolor{beamer@blendedblue}{rgb}{0.137,0.466,0.741}

\setbeamercolor{structure}{fg=beamer@blendedblue}
\setbeamercolor{titlelike}{parent=structure}
\setbeamercolor{frametitle}{fg=black}
\setbeamercolor{title}{fg=black}
\setbeamercolor{item}{fg=black}

\mode
<all>

%commands
\newcommand{\U}{\mathscr{U}}
\newcommand{\R}{\mathbb{R}}

\theoremstyle{definition}
\newtheorem{question}{Question}[section]

\theoremstyle{definition}
\newtheorem{answer}{Answer}[section]


\setbeamercovered{invisible}

\title{J\' onsson\,-Tarski Theorem}
\author{Arka Ghosh}
\date{}

\begin{document}

\begin{frame}\label{frame:titlepage}
\titlepage
\end{frame}

%%%%%%%%%%%%%%%%%%%%%%%%%%%%%%%%%%%%%%%%%%%%%%%%%%%%%%%%%%%%%%%%%%%%%%%%%%%%%%%%%

\begin{frame}\label{frame:modal similarity type}
\begin{definition}[Modal similarity type]\label{defn:modal similarity type}
A modal similarity type $\tau$ is a pair $(O,\rho)$ where $O$ is a non-empty set, 
and $\rho$ is a function $O\to\mathbb{N}$. An element $\Delta \in O$ is called a modal
operator of arity $\rho(\Delta)$.
\end{definition}
\end{frame}

%%%%%%%%%%%%%%%%%%%%%%%%%%%%%%%%%%%%%%%%%%%%%%%%%%%%%%%%%%%%%%%%%%%%%%%%%%%%%%%%%%%%

\begin{frame}\label{frame:boolean algebra}
\begin{definition}[Boolean algebra]\label{defn:boolean algebra}
Let $A$ be a non-empty set, $0 \in A$, $+$ be a binary operator and $-$ be an unary operator
on $A$. Define $1 := 0$ and let $x\cdot y$ be the abbreviation for $-((-x)+(-y))$.\smallskip

$\mathcal{K} = (A,+,-,0)$ is a called a boolean algebra if the following hold
for $x, y, z \in A$:
\begin{multicols}{2}
\begin{enumerate}
\item[$\triangleright$] $-(-x) = x$
\item[$\triangleright$] $x + 0 = x$
\item[$\triangleright$] $x + 1 = 1$
\item[$\triangleright$] $x + x = x$
\item[$\triangleright$] $x + (y + z) = (x + y) + z$
\item[$\triangleright$] $x + y = y + x$
\item[$\triangleright$] $x + (-x) = 1$
\item[$\triangleright$] $x + (y\cdot z) = (x + y) \cdot (x + z)$
\item[]
\item[$\triangleright$] $x \cdot 0 = 0$
\item[$\triangleright$] $x \cdot 1 = x$
\item[$\triangleright$] $x \cdot x = x$
\item[$\triangleright$] $x \cdot (y \cdot z) = (x \cdot y) \cdot z$
\item[$\triangleright$] $x \cdot y = y \cdot x$
\item[$\triangleright$] $x \cdot (-x) = 1$
\item[$\triangleright$] $x \cdot (y + z) = (x \cdot y) + (x \cdot z)$
\end{enumerate}
\end{multicols}
\end{definition}
\end{frame}

%%%%%%%%%%%%%%%%%%%%%%%%%%%%%%%%%%%%%%%%%%%%%%%%%%%%%%%%%%%%%%%%%%%%%%%%%%%%%%%%%%%

\begin{frame}\label{frame: filters and ultrafilters}

\begin{definition}[Filter]\label{defn:filter}
Given a boolean algebra $\mathcal{K} = (A,+,-,0)$, $F\subseteq A$ is called a filter if:
\begin{enumerate}
\item[$\triangleright$] $1 \in F$,
\item[$\triangleright$] If $a,b \in F$ then $a \cdot b \in F$,
\item[$\triangleright$] If $a \in F$ and $a \leq b$ then $b \in F$.
\end{enumerate}
$F$ is called proper if $0 \notin F$, or, equivalently, if $F$ is a strict subset of $A$.
\end{definition}

\visible<2->{
\begin{definition}[Ultrafilter]\label{defn:ultrafilter}
A filter $F$ is called an ultrafilter if for every $a \in A$, either $a$ or $-a$ 
belongs to $F$. We will denote the collection of ultrafilters of $\mathcal{K}$ as
$\mathbb{UF}(\mathcal{K})$. 
\end{definition}
}

\end{frame}

\begin{frame}\label{frame:examples of filters}

\visible<1->{
\begin{example}\label{example:filter generated by an element}
Given $a \in A$, the set $a\uparrow := \{b \in A\ |\ a \leq b\}$ is a filter.
\end{example}
}

\visible<2->{
\begin{example}\label{example:filter generated by a subset}
Given $D \subseteq A$ the filter generated by $D$ is defined as, 
 $$F_D := \{a \in A\ |\ \exists d_0,\dots,d_n \in D \text{ s.t. } 
   d_0\cdot \dots \cdot d_n \leq a \}.$$
This is also the smallest filter containing $D$.   
\end{example}
}

\end{frame}

%%%%%%%%%%%%%%%%%%%%%%%%%%%%%%%%%%%%%%%%%%%%%%%%%%%%%%%%%%%%%%%%%%%%%%%%%%%%%%%%%%%%%%%%%%%%%%

\begin{frame}\label{frame:finite meet property}

\visible<1->{
\begin{definition}[Finite meet property]\label{defn:finite meet property}
$D \subseteq A$ is said to have finite meet property if there is no finite subset 
$\{d_0,\dots,d_n\}$ of $D$ such that $d_0\cdot \dots \cdot d_n = 0$. 
\end{definition}
}

\visible<2->{
\begin{lemma}\label{lemma:when generated filter is proper}
Given $D \subseteq A$, $F_D$ is proper if and only if $D$ has the finite meet property
\end{lemma}
}

\end{frame}

%%%%%%%%%%%%%%%%%%%%%%%%%%%%%%%%%%%%%%%%%%%%%%%%%%%%%%%%%%%%%%%%%%%%%%%%%%%%%%%%%%%%%%%%%%%%%%

\begin{frame}\label{frame:prp of ultrafilters}

\visible<1->{
\begin{lemma}\label{lemma:ultrafilters are prime wrt +}
For and ultrafilter $F$, $a + b \in F$ iff $a \in F$ or $b \in F$.
\end{lemma}
}

\visible<2->{
\begin{theorem}\label{thm:ultrafilters are maximal}
Ultrafilters are also maximal filters. That is, if $F$ is an ultrafilter and $L$ is a proper
filter such that $F \subseteq L$ then $F = L$.
\end{theorem}
}

\end{frame}

%%%%%%%%%%%%%%%%%%%%%%%%%%%%%%%%%%%%%%%%%%%%%%%%%%%%%%%%%%%%%%%%%%%%%%%%%%%%%%%%%%%%%%%%%%%%%%

\end{document}
\documentclass[14pt,compress]{beamer}
\usepackage[utf8]{inputenc}
\usepackage[T1]{fontenc}
\usepackage{lmodern}
\usepackage{tikz}
\usetikzlibrary{patterns}
\usepackage{tikz-cd}
%\tikzset{commutative diagrams/arrow style=math font}

% Beamer commands
\title{Introduction to Type Theory and Lambda Calculus}
\date{}
%\setbeamertemplate{navigation symbols}{}
%\setbeamertemplate{blocks}[rounded][shadow=false]
%\usecolortheme{orchid}
% TikZ options
\usetikzlibrary{arrows,shapes}
\useoutertheme{split}
\useoutertheme[footline=authortitle]{miniframes}
%\useinnertheme{circles}
%\usecolortheme{whale}
%\usecolortheme{orchid}

\definecolor{beamer@blendedblue}{rgb}{0.3,0.5,0.8}
\definecolor{beamer@myviolet}{rgb}{0.7,0.2,0.5}
\definecolor{beamer@deepblue}{rgb}{0.5,0.5,0.7}
\definecolor{beamer@lightgray}{rgb}{0.5,0.7,0.5}
\definecolor{beamer@mybrown}{rgb}{0.3,0.3,0.2}
\definecolor{beamer@mathtext}{rgb}{0.9,0.5,0.4}
\definecolor{beamer@header}{rgb}{0.4,0.1,0.1}

\setbeamercolor{background canvas}{fg=white, bg=black}
\setbeamercolor{normal text}{fg=beamer@lightgray,bg=black}
\setbeamercolor{alerted text}{fg=red}
\setbeamercolor{example text}{fg=green!50!black}
\setbeamercolor{miniframes}{fg=red,bg=white}
\setbeamercolor{structure}{fg=beamer@deepblue}
\setbeamercolor{titlelike}{fg=magenta}
\setbeamercolor{frametitle}{fg=beamer@myviolet}
\setbeamercolor{title}{fg=beamer@myviolet}
\setbeamercolor{item}{fg=beamer@mybrown}
\setbeamercolor{section in head/foot}{fg=white,bg=beamer@header}

\setbeamerfont{framesubtitle}{size=10pt}


\mode
<all>

%\setbeamercovered{invisible}
\begin{document}

%%%%%%%%%%%%%%%%%%%%%%%%%%%%%%%%%%%%%%%%%%%%%%%%%%%%%%%%%%%%%%%%%%%%%%%%%%%%%%%%%%%%%%%%%%

\begin{frame}\label{frame : titlepage}
\titlepage
\end{frame}

%%%%%%%%%%%%%%%%%%%%%%%%%%%%%%%%%%%%%%%%%%%%%%%%%%%%%%%%%%%%%%%%%%%%%%%%%%%%%%%%%%%%%%%%%%

\begin{frame}\label{frame : what is a type theory}
\frametitle{What is a Type Theory?}

\begin{enumerate}
\pause
\item[$\blacktriangleright$] A (programming)language.\only<4->{\textit{\textcolor{beamer@mathtext}{
$\leftarrow$ Let's see this first}}}
\pause
\item[$\blacktriangleright$] A foundational system.
\end{enumerate}

\end{frame}

%%%%%%%%%%%%%%%%%%%%%%%%%%%%%%%%%%%%%%%%%%%%%%%%%%%%%%%%%%%%%%%%%%%%%%%%%%%%%%%%%%%%%%%%%%

\begin{frame}\label{frame : what is a language}
\frametitle{What is a language?}

\begin{enumerate}
\pause
\item[$\blacktriangleright$] Alphabets and symbols.
\pause
\item[$\blacktriangleright$] Words(combination of alphabets).
\pause
\item[$\blacktriangleright$] Semi-sentences\\(combination of words and symbols).
\pause
\item[$\blacktriangleright$] Grammer(which semi-sentences are sentences).
\pause
\item[$\blacktriangleright$] Informal meaning.
\end{enumerate}

\end{frame}

%%%%%%%%%%%%%%%%%%%%%%%%%%%%%%%%%%%%%%%%%%%%%%%%%%%%%%%%%%%%%%%%%%%%%%%%%%%%%%%%%%%%%%%%%%

\begin{frame}\label{frame : purpose of a language}
\frametitle{What is a language?}

\begin{enumerate}
\item[$\blacktriangleright$] Sometimes languages are designed for specfic purposes.
\pause
\item[$\blacktriangleright$] One of the earlier motivation to create a type theory was to
model computation. 
\end{enumerate}

\end{frame}

%%%%%%%%%%%%%%%%%%%%%%%%%%%%%%%%%%%%%%%%%%%%%%%%%%%%%%%%%%%%%%%%%%%%%%%%%%%%%%%%%%%%%%%%%%

\begin{frame}\label{frame : what is computation}

\frametitle{What is Computation?}
\begin{enumerate}
\pause
\item[$\blacktriangleright$] Change of state(Turing Machines).
\pause
\item[$\blacktriangleright$] Function application(lambda calculus). 
\end{enumerate}
\pause
\textcolor{beamer@mathtext}{These two modes of computation are equivalent(Church-Turing thesis).}

\end{frame}

%%%%%%%%%%%%%%%%%%%%%%%%%%%%%%%%%%%%%%%%%%%%%%%%%%%%%%%%%%%%%%%%%%%%%%%%%%%%%%%%%%%%%%%%%%

\begin{frame}\label{frame : terms of lambda calculus}

\frametitle{Untyped Lambda Calculus}

\pause
\begin{block}{Terms}
\[ M := x \in Var\ |\ (M_1\ M_2)\ | \lambda x . M \]
\end{block}
\end{frame}

%%%%%%%%%%%%%%%%%%%%%%%%%%%%%%%%%%%%%%%%%%%%%%%%%%%%%%%%%%%%%%%%%%%%%%%%%%%%%%%%%%%%%%%%%%

\begin{frame}\label{frame : rules of lambda calculus}

\frametitle{Untyped Lambda Calculus}
\begin{block}{Evaluation($\beta$-reduction)}
\[ ((\lambda x.M)\ y) \to M[x/y] \]
\[ \text{If }M_1 \to M_2 \text{ then }(P\ M_1) \to (P\ M_2) \]
\[ \text{If }M_1 \to M_2 \text{ then }(M_1\ N) \to (M_2\ N) \]
\[ \text{If }M_1 \to M_2 \text{ then }(\lambda x.M_1) \to (\lambda M. t_2) \]
\end{block}
\pause
\begin{block}{$\alpha$-conversion}
Terms are considered equivalent upto renaming of bound variables.
$\lambda x.M \equiv \lambda y.M[x/y]$.
\end{block}
\end{frame}

%%%%%%%%%%%%%%%%%%%%%%%%%%%%%%%%%%%%%%%%%%%%%%%%%%%%%%%%%%%%%%%%%%%%%%%%%%%%%%%%%%%%%%%%%%

\begin{frame}\label{frame : examples1 of terms}
\frametitle{Untyped Lambda Calculus}

\begin{block}{Examples of terms}
\pause \[ I := \lambda x.x \]
\pause \[ c_0 := \lambda f.\lambda x. x \]
\pause \[ c_{n+1} := \lambda f.\lambda x.(f\ ((c_n\ f)\ x)) \]
\pause \[ succ := \lambda n. \lambda f. \lambda x.(f\ ((n\ f)\ x)) \]
\end{block}
\end{frame}

%%%%%%%%%%%%%%%%%%%%%%%%%%%%%%%%%%%%%%%%%%%%%%%%%%%%%%%%%%%%%%%%%%%%%%%%%%%%%%%%%%%%%%%%%%

\begin{frame}\label{frame : examples2 of terms}
\frametitle{Untyped Lambda Calculus}

\begin{block}{Examples of terms}
\pause \[ K := \lambda x.\lambda y. x \]
\pause \[ L := \lambda x.\lambda y. y \]
\pause \[ S := \lambda x. \lambda y. \lambda z.((x\ z)(y\ z))\]
\pause \[ Y := \lambda x.(x\ x) \]
\end{block}
\end{frame}

%%%%%%%%%%%%%%%%%%%%%%%%%%%%%%%%%%%%%%%%%%%%%%%%%%%%%%%%%%%%%%%%%%%%%%%%%%%%%%%%%%%%%%%%%%
\begin{frame}\label{frame : notation}
\frametitle{Untyped Lambda Calculus}

\begin{block}{Notation}
From now on we will sometimes write $x_1 \dots x_n$ to mean $(\dots((x_1\ x_2)\ x_3) \dots \ x_n)$.
\end{block} 
\end{frame}

%%%%%%%%%%%%%%%%%%%%%%%%%%%%%%%%%%%%%%%%%%%%%%%%%%%%%%%%%%%%%%%%%%%%%%%%%%%%%%%%%%%%%%%%%%
\begin{frame}\label{frame : example of evaluation}
\frametitle{Untyped Lambda Calculus}

\begin{block}{Examples of evaluations}
\pause
\[ (I\ x) \to x \]
\pause
\[ (succ\ c_n) \to c_{n+1} \]
\pause
\[ SKpq \to Kq(pq) \to q\]
\pause
\[ (\lambda x.xx)(\lambda x.xx) \to (\lambda x.xx)(\lambda x.xx) \]
\end{block}

\end{frame}

%%%%%%%%%%%%%%%%%%%%%%%%%%%%%%%%%%%%%%%%%%%%%%%%%%%%%%%%%%%%%%%%%%%%%%%%%%%%%%%%%%%%%%%%%%
\begin{frame}\label{frame : free and bound variables}
\frametitle{Untyped Lambda Calculus}
\begin{block}{Free and bound variables}
The collection of free variables in a term $M$, written as $FV(M)$, is defined as:
\textcolor{beamer@mathtext}{
\[ FV(x) := \{x\} \]
\[ FV((M N)) := FV(M)\cup FV(N) \]
\[ FV(\lambda x.M) := FV(M)\setminus \{x\} \]}
\pause
A variable occuring in $M$ is bound if it is not free.
\end{block}


\end{frame}

%%%%%%%%%%%%%%%%%%%%%%%%%%%%%%%%%%%%%%%%%%%%%%%%%%%%%%%%%%%%%%%%%%%%%%%%%%%%%%%%%%%%%%%%%%
\begin{frame}\label{frame : variable capture}
\frametitle{Untyped Lambda Calculus}

\begin{block}{Some coventions}
Problems like the following can occur while evaluating a term
\textcolor{beamer@mathtext}{
\[ (\lambda x.\lambda y.x y) y z \to (\lambda y.y y) z \to z \]}
\pause
To take care of this, we rename the bound variables accordingly so that in $(M\ N)$,
free variables of $N$ don't coincide with bound variables of $M$.
\end{block}

\end{frame}
%%%%%%%%%%%%%%%%%%%%%%%%%%%%%%%%%%%%%%%%%%%%%%%%%%%%%%%%%%%%%%%%%%%%%%%%%%%%%%%%%%%%%%%%%%
\begin{frame}[fragile]\label{frame : Church Rosser Theorem}
\frametitle{Untyped Lambda Calculus}

\begin{theorem}[Church-Rosser]
Let $\twoheadrightarrow$ be the reflexive transitive closure of $\to$.
If $P \twoheadrightarrow Q_1$ and $P \twoheadrightarrow Q_2$,
then there is an $R$ such that $Q_1 \twoheadrightarrow R$ and $Q_2 \twoheadrightarrow R$.
\end{theorem}
\color{beamer@mathtext}{
\begin{equation*}
\begin{tikzcd}
& P \arrow[rd,twoheadrightarrow] \arrow[ld,twoheadrightarrow] & \\
Q_1 \arrow[rd,dashed,twoheadrightarrow] & & Q_2 \arrow[ld,dashed,twoheadrightarrow] \\
& R &
\end{tikzcd}
\end{equation*}}
\end{frame}
%%%%%%%%%%%%%%%%%%%%%%%%%%%%%%%%%%%%%%%%%%%%%%%%%%%%%%%%%%%%%%%%%%%%%%%%%%%%%%%%%%%%%%%%%%
\begin{frame}\label{frame : power of lambda}
\frametitle{Untyped Lambda Calculus}
\begin{enumerate}
\pause
\item[$\blacktriangleright$] We have already seen Church-encodings of natural numbers.
\pause
\item[$\blacktriangleright$] Similarly we can define addition, multiplication and exponentiation.
\pause
\item[$\blacktriangleright$] We can also define datatypes such as lists and trees, and operations on them.
\pause
\item[$\blacktriangleright$] In fact using lambda calculus we can define any function
which can be computed using a Turing Machine.
\end{enumerate}
\end{frame}
\end{document}


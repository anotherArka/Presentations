\documentclass[10pt]{beamer}
\usepackage{listings}
%\usetheme{Boadilla}
%\usecolortheme{beaver}

\useoutertheme{smoothbars}

\usepackage{lmodern}
\usepackage[scale=2]{ccicons}

\usepackage[T1]{fontenc}
\usepackage{multicol}
\usepackage[latin9]{inputenc}
\usepackage{graphicx}
\usepackage{array}
%\usepackage{enumitem}
\usepackage{amsthm,amsmath,latexsym,amssymb,amsmath}
\usepackage{mathrsfs,graphicx,xargs,etoolbox}


\DeclareOptionBeamer{compress}{\beamer@compresstrue}
\ProcessOptionsBeamer

\mode<presentation>

\useoutertheme{split}
\useoutertheme[footline=authortitle]{miniframes}
\useinnertheme{circles}
\usecolortheme{whale}
\usecolortheme{orchid}

\definecolor{beamer@blendedblue}{rgb}{0.3,0.5,0.8}

\setbeamercolor{structure}{fg=beamer@blendedblue}
\setbeamercolor{titlelike}{parent=structure}
\setbeamercolor{frametitle}{fg=black}
\setbeamercolor{title}{fg=black}
\setbeamercolor{item}{fg=black}

\mode
<all>

%commands
\newcommand{\U}{\mathscr{U}}
\newcommand{\R}{\mathbb{R}}

\theoremstyle{definition}
\newtheorem{question}{Question}[section]

\theoremstyle{definition}
\newtheorem{answer}{Answer}[section]


\setbeamercovered{invisible}

\title{The Yoneda Lemma}
\author{Arka Ghosh}
\date{}

\begin{document}

\begin{frame}\label{frame: titlepage}
\titlepage
\end{frame}

%%%%%%%%%%%%%%%%%%%%%%%%%%%%%%%%%%%%%%%%%%%%%%%%%%%%%%%%%%%%%%%%%%%%%%%%%%%%%%%%%
\section{Representable Functors}
\begin{frame}\label{frame: representable functors}
\frametitle{Representable Functors}

\begin{definition}[Representation of a functor]
\begin{enumerate}

\item[$\blacktriangleright$]
A representation of a covariant functor $F : C \to Set$ consists of an object $c \in C$ 
along with a natural isomorphism $\alpha : Hom(c,-) \cong F$.\smallskip

\item[$\blacktriangleright$] 
Dually, A representation of a contravariant functor $F : C \to Set$ consists of an object 
$c \in C$ along with a natural isomorphism $\alpha : Hom(-,c) \cong F$.\smallskip

\item[$\blacktriangleright$]
A funtor is representable if it has a representation

\item[$\blacktriangleright$]
For issues with size, we need $C$ to be locally small.

\end{enumerate}

\end{definition}
\end{frame}

%%%%%%%%%%%%%%%%%%%%%%%%%%%%%%%%%%%%%%%%%%%%%%%%%%%%%%%%%%%%%%%%%%%%%%%%%%%%%%%%%%%%

\begin{frame}\label{frame : eg of rep func}
\frametitle{Examples}

\begin{block}{Covariant representable functors}
\begin{enumerate}

\item[$\blacktriangleright$] The identity functor $id : Set \to Set$ is represented by
the singleton set $\mathbf{1}$. That is, for a set $X$
$Hom(\mathbf{1}, X) \cong X$.

\item[$\blacktriangleright$] The forgetful functor $U : Group \to Set$ is represented by
$\mathbb{Z}$.

\item[$\blacktriangleright$] The forgetful functor $U : Top \to Set$ is represented by the
singleton space.

\item[$\blacktriangleright$] The functor $Path : Top \to Set$ which maps a topological space
$X$ to its set of paths is represented by the unit interval $I$.
\end{enumerate}

\end{block}

\end{frame}


\end{document}
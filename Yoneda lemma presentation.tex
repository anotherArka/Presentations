\documentclass[10pt,compress]{beamer}
\usepackage[utf8]{inputenc}
\usepackage[T1]{fontenc}
\usepackage{lmodern}
\usepackage{tikz}
\usetikzlibrary{patterns}
\usepackage{tikz-cd}
%\tikzset{commutative diagrams/arrow style=math font}

% Beamer commands
\title{The Yoneda Lemma}
%\setbeamertemplate{navigation symbols}{}
%\setbeamertemplate{blocks}[rounded][shadow=false]
%\usecolortheme{orchid}
% TikZ options
\usetikzlibrary{arrows,shapes}
\useoutertheme{split}
\useoutertheme[footline=authortitle]{miniframes}
%\useinnertheme{circles}
%\usecolortheme{whale}
%\usecolortheme{orchid}

\definecolor{beamer@blendedblue}{rgb}{0.3,0.5,0.8}
\definecolor{beamer@myviolet}{rgb}{0.7,0.2,0.5}
\definecolor{beamer@deepblue}{rgb}{0.5,0.5,0.7}
\definecolor{beamer@lightgray}{rgb}{0.5,0.7,0.5}
\definecolor{beamer@mybrown}{rgb}{0.3,0.3,0.2}
\definecolor{beamer@mathtext}{rgb}{0.9,0.5,0.4}
\definecolor{beamer@header}{rgb}{0.4,0.1,0.1}

\setbeamercolor{background canvas}{fg=white, bg=black}
\setbeamercolor{normal text}{fg=beamer@lightgray,bg=black}
\setbeamercolor{alerted text}{fg=red}
\setbeamercolor{example text}{fg=green!50!black}
\setbeamercolor{miniframes}{fg=red,bg=white}
\setbeamercolor{structure}{fg=beamer@deepblue}
\setbeamercolor{titlelike}{fg=magenta}
\setbeamercolor{frametitle}{fg=beamer@myviolet}
\setbeamercolor{title}{fg=beamer@myviolet}
\setbeamercolor{item}{fg=beamer@mybrown}
\setbeamercolor{section in head/foot}{fg=white,bg=beamer@header}

\setbeamerfont{framesubtitle}{size=10pt}


\mode
<all>

%\setbeamercovered{invisible}
\begin{document}

\begin{frame}\label{frame: titlepage}
\titlepage
\end{frame}

%%%%%%%%%%%%%%%%%%%%%%%%%%%%%%%%%%%%%%%%%%%%%%%%%%%%%%%%%%%%%%%%%%%%%%%%%%%%%%%%%%%%%%%%%%
\section{Representable Functors}
\begin{frame}\label{frame: representable functors}
\frametitle{Representable Functors}

\begin{definition}[Representation of a functor]
\begin{enumerate}

\item[$\blacktriangleright$]
A representation of a covariant functor $F : C \to Set$ consists of an object $c \in C$ 
along with a natural isomorphism $\alpha : Hom(c,-) \cong F$.\smallskip

\item[$\blacktriangleright$] 
Dually, A representation of a contravariant functor $F : C \to Set$ consists of an object 
$c \in C$ along with a natural isomorphism $\alpha : Hom(-,c) \cong F$.\smallskip

\item[$\blacktriangleright$]
A funtor is representable if it has a representation.

\item[$\blacktriangleright$]
For issues with size, we need $C$ to be locally small.

\end{enumerate}

\end{definition}
\end{frame}

%%%%%%%%%%%%%%%%%%%%%%%%%%%%%%%%%%%%%%%%%%%%%%%%%%%%%%%%%%%%%%%%%%%%%%%%%%%%%%%%%%%%%%%%%%%%%%
\begin{frame}\label{frame : eg of cov rep func}
\frametitle{Examples}

\begin{block}{Covariant representable functors}
\begin{enumerate}

\item[$\blacktriangleright$] The identity functor $id : Set \to Set$ is represented by
the singleton set $\mathbf{1}$. That is, for a set $X$
$Hom(\mathbf{1}, X) \cong X$.

\item[$\blacktriangleright$] The forgetful functor $U : Group \to Set$ is represented by
$\mathbb{Z}$.

\item[$\blacktriangleright$] The forgetful functor $U : Top \to Set$ is represented by the
singleton space.

\item[$\blacktriangleright$] The functor $Path : Top \to Set$ which maps a topological space
$X$ to its set of paths is represented by the unit interval $I$.
\end{enumerate}

\end{block}

\end{frame}

%%%%%%%%%%%%%%%%%%%%%%%%%%%%%%%%%%%%%%%%%%%%%%%%%%%%%%%%%%%%%%%%%%%%%%%%%%%%%%%%%%%%%%%%%%%%%%%

\begin{frame}\label{frame : eg of contra rep func}

\frametitle{Examples}

\begin{block}{Contravariant representable functors}
\begin{enumerate}

\item[$\blacktriangleright$] 
The contravariant power set functor $P : Set^{op} \to Set$ is represented by the set
$\mathbf{2} = \{\top, \bot\}$. That is $Hom(X,\mathbf{2}) \cong P(X)$.

\item[$\blacktriangleright$] 
The functor $\mathcal{O} : Top^{op} \to Set$ which maps a topological space to its
set of open subsets is represented by the Sierpienski space $S$.
\[ Give diagram\]

\item[$\blacktriangleright$]
The functor $D : Vect_K^{op} \to Set$ which maps a vector space to its dual is represented by
$K$, considered as a vector space on itself.

\end{enumerate}

\end{block}

\end{frame}

%%%%%%%%%%%%%%%%%%%%%%%%%%%%%%%%%%%%%%%%%%%%%%%%%%%%%%%%%%%%%%%%%%%%%%%%%%%%%%%%%%%%%%%%%%%%%%%

\section{The Yoneda Lemma}
\begin{frame}

\begin{theorem}[The Yoneda lemma]
\only<1>{
For a locally small category $C$ and a covariant functor $F : C \to Set$, 
there is a bijection
\[ Y_c : Nat(Hom(c,-), F) \cong Fc .\]
It maps a natural transformation $\alpha : Hom(c,-) \implies F$ to the element 
$\alpha_c(id_c)$, and is natural in both $c$ and $F$.\smallskip
}

\only<2>{
Dually, for a locally small category $C$ and a contravariant functor $F : C \to Set$, 
there is a bijection
\[ Y_c : Nat(Hom(-,c), F) \cong Fc .\]
It maps a natural transformation $\alpha : Hom(-,c) \implies F$ to the element 
$\alpha_c(id_c)$, and is natural in both $c$ and $F$.\smallskip
}
\end{theorem}


\end{frame}

%%%%%%%%%%%%%%%%%%%%%%%%%%%%%%%%%%%%%%%%%%%%%%%%%%%%%%%%%%%%%%%%%%%%%%%%%%%%%%%%%%%%%%%%%%%%%%%

\section{The Yoneda Embedding}
\begin{frame}[fragile]

\frametitle{The Yoneda Embedding}

The following define full and faithful embeddings:

\color{beamer@mathtext}{
\begin{equation*}
\begin{tikzcd}
	C \arrow[r, "Y", hook] & Set^{C^{op}}\\
    c \arrow[r, mapsto] \arrow[d, "f" '] & Hom(-,c) \arrow[d, "f_*"] \\
    d \arrow[r, mapsto] & Hom(-,d)	 
\end{tikzcd} \hspace{20pt}
\begin{tikzcd}
	C^{op} \arrow[r, "Y", hook] & Set^{C}\\
    c \arrow[r, mapsto] \arrow[d, "f" '] & Hom(c,-)  \\
    d \arrow[r, mapsto] & Hom(d,-)	\arrow[u, "f_*" '] 
\end{tikzcd}
\end{equation*}
}
\end{frame}

\section{Consequences}

\begin{frame}

\begin{block}{An alternate proof of Cayley's theorem}
Write the proof
\end{block}	
	
\end{frame}

%%%%%%%%%%%%%%%%%%%%%%%%%%%%%%%%%%%%%%%%%%%%%%%%%%%%%%%%%%%%%%%%%%%%%%%%%%%%%%%%%%%%%%%%%%%%%%

\begin{frame}
\frametitle{Some nice properties of $\widehat{C} = Set^{C^{op}}$}

\begin{enumerate}

\item[$\blacktriangleright$]
$\widehat{C}$ has small limits and co-limits. For a small diagram $F : I \to \widehat{C}$ 

\textcolor{beamer@mathtext}{
\[ (\lim_{\to} F(i))(U) = (\lim_{\to}F(i)(U)) \]
}


Specifically $\widehat{C}$ has products and co-products indexed over a set.

\item[$\blacktriangleright$]
$\widehat{C}$ has exponentials when $C$ is small. For $A, B : C^{op} \to Set$ we have

\textcolor{beamer@mathtext}{
\[ B^A(I) = Nat(Y(I) \times A, B) \]
\[ B^A(u) = Nat(Y(u) \times A, B) \]
}

\end{enumerate}




\end{frame}

\end{document}